\documentclass[11pt]{article}
\usepackage[english]{babel}
%\usepackage{blindtext}
\usepackage{lineno}
\usepackage{fullpage}
\usepackage{amsmath, amssymb, graphicx, color, array, appendix}
%\renewcommand*\familydefault{\sfdefault} %% Only if the base font of the document is to be sans serif
\usepackage[T1]{fontenc}
\usepackage[utf8]{inputenc}
%\usepackage{sistyle}
\usepackage{setspace}
\usepackage{soul}
\usepackage{hyperref}
\usepackage{verbatim}
\usepackage{fancyhdr}
\fancyhf{} % sets both header and footer to nothing
\renewcommand{\headrulewidth}{0pt}
\usepackage{lastpage}
\pagestyle{fancy}
\usepackage{times}
\usepackage{changepage}
\usepackage{footnote}
%\cfoot{Page \thepage\ of \pageref{LastPage}}
\usepackage{calc}
\onehalfspacing

\begin{document}
\setlength{\parindent}{1cm}

\exhyphenpenalty=10000
\interfootnotelinepenalty=10000
\frenchspacing
\vspace{-0.5in}

\begin{center}
\Huge
\emph{Intra Vires} Research\\
\Large
\textsc{Testing Scenario}
\end{center}
 \normalsize

\begin{center}
\vspace{-0.25in}
\line(1,0){361}


\normalsize
\textsc{David Pardy \& Stephen Huang}\\
\textsc{July, 2013}\\
\textsc{Confidential} -- Do not discuss outside this session.
\end{center}


You are a law student working on a memo. It's 9:30PM and you're in the library. You are in a hurry because you want to go out with your friends for the weekly pub night. The memo is due tomorrow. You only have 4 citations left.

You recall that your friend Antonio Lamer told you about the website ``intra dash vires dot com'' that automates the citation process. You decide to try it out, especially because you have some tricky bits that you're not sure how to cite and you don't have a McGill Guide handy.\\

\begin{enumerate}
\item Your memo revolves around \emph{Adams v Thompson, Berwick, Pratt, \& Partners} (1987). It's available on CanLII. You will be citing it multiple times in your memo, so you want to write a short form. You don't need to pinpoint it for this first citation, but you do need to say where you will be citing it in the future. It was also written by now Chief Justice of Canada Beverly ``Bev'' McLachlin, so you'll want to make sure the reader knows that.
\item \emph{Dunsmuir} (2008) is a famous Supreme Court of Canada case on administrative law. It's available on CanLII. You are including a summary in your memo, including procedural posture (i.e. you want to be sure to say that it affirmed the New Brunswick Court of Appeal case 2006 NBCA 27). In this particular summary you are focusing on Justice Binnie's passage in paragraph 132. You will be citing \emph{Dunsmuir} again, so you want to create a short form.
\item In \emph{R v Sparrow} (1990) at page 1103, Chief Justice of Canada Dickson cited a passage from the British Columbia Supreme Court in \emph{Pasco v Canadian National Railway Co} ([1986] 1 CNLR 35 at 37 (available on CanLII) (BC SC)). They are both on CanLII. In your citation, you want to let the reader know that Dickson CJC adopted the phrase, but that he cited the original case. Both of these cases are available on CanLII. You won't be citing this case again in your memo.%\emph{R v Sparrow}, [1990] 1 SCR 1075 at 1103, 70 DLR (4th) 385, Dickson CJC [\emph{Sparrow}] citing \emph{Pasco v Canadian National Railway Co}, [1986] 1 CNLR 35 at 37 (available on CanLII) (BC SC).
\item Finally, you want to throw in a citation to the \emph{Securities Act Reference} from 2011. It's available on CanLII. You don't care about the author of the judgement or the procedural posture, and you won't cite it again.
\end{enumerate}








\end{document}