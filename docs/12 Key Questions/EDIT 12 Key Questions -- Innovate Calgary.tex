\documentclass[11pt]{article}
\usepackage[english]{babel}
%\usepackage{blindtext}
\usepackage{lineno}
%\usepackage{fullpage}
\usepackage{amsmath, amssymb, graphicx, color, array, appendix}
%\renewcommand*\familydefault{\sfdefault} %% Only if the base font of the document is to be sans serif
\usepackage[T1]{fontenc}
\usepackage[utf8]{inputenc}
%\usepackage{sistyle}
\usepackage{setspace}
\usepackage{soul}
\usepackage{hyperref}
\usepackage{verbatim}
\usepackage{fancyhdr}
\fancyhf{} % sets both header and footer to nothing
\renewcommand{\headrulewidth}{0pt}
\usepackage{lastpage}
\pagestyle{fancy}
\usepackage{times}
\usepackage{changepage}
\usepackage{footnote}
\cfoot{Page \thepage\ of \pageref{LastPage}}
\usepackage{calc}
\doublespacing



\newcolumntype{L}[1]{>{\raggedright\let\newline\\\arraybackslash\hspace{0pt}}m{#1}}
\newcolumntype{C}[1]{>{\centering\let\newline\\\arraybackslash\hspace{0pt}}m{#1}}
\newcolumntype{R}[1]{>{\raggedleft\let\newline\\\arraybackslash\hspace{0pt}}m{#1}}

\begin{document}
\setlength{\parindent}{1cm}

\exhyphenpenalty=10000
\interfootnotelinepenalty=10000
\frenchspacing

\begin{center}
\Huge
\emph{Intra Vires} Research
\end{center}
 \normalsize

\begin{center}
\vspace{-0.5in}
\line(1,0){361}


\normalsize
\textsc{David Pardy \& Stephen Huang}\\
\textsc{June 19, 2013}\\
CONFIDENTIAL -- Intended for Innovate Calgary only
\end{center}

\section*{Introduction}


Law firms make their business by advising clients on legal issues, representing them in court, and providing documents to them.  Big law firms charge clients from \$700 to \$1200 per hour, plus research fees ranging from \$50 to over \$2000 per question. Learning how to research efficiently is paramount, and law schools have long made it a large part of the curriculum.



%From personal experience, we believe that the availability of the technology to the entire law community is severely lacking. Law firms use old computers that frequently crash, with outdated filing systems and no tools in place to research more efficiently. There are no problems law With so much room for improvement, the question is not \emph{what} can we do, but \emph{how much} we can do with the time and resources we have.

\section*{The Problem and the Solution}

Incoming law students do not know how to do legal research, and there is not much time to learn because the curriculum is so intense. Shortcuts are taken wherever possible, without compromising quality. The problem is that there are no good shortcuts.

Legal research involves ploughing through enormous databases and old books to find the state of the law (a "precedent") on discrete, often tangential issues. It can be very confusing. Even keeping track of what cases you have read, and where they were found, is annoying. We see the potential for software to help with knowledge management through the use of precedent trees and a highly interactive personal precedent bank. 

After finding and analyzing cases, one must then cite the precedents correctly. Even for experts, citing can take about five hours. The Canadian standard for citations is published in a manual called the Canadian Guide to Uniform Legal Citation, a dense volume of 600 pages (the ``McGill Guide''). Learning to use it properly can take months. Many law students never learn how to use it, and their marks suffer for it. Our product will automate the process, turning hours spent citing into mere minutes.

\section*{The Market}

The market for an automated citation engine is large and stable. It includes lawyers who do legal research, law students, and academics. In a Bloomberg Law interview with 30-year law firm consultant and author Richard Susskind on May 2, 2013, he identified the influx of technology to law firms as one of the three key opportunities for profitable change in the field. The other two were reducing costs to clients, and using non-lawyers in the delivery of legal services. All three categories are encapsulated in our proposed product. 

%\subsection*{Concerns}
%Legal service sector growth generally mirrors the local economy, with a lag of a few years. Recent concerns over a drop in demand for articling students is a result of the 2008 financial crisis, though growth is returning to the sector. Law schools have been very consistent with the admission numbers of law students across all of Canada.
A major hurdle to accessing the legal firms is the rigidity of most law firms' software choices. After speaking to director of knowledge management at Norton Rose Fulbright LLP, we discovered that getting our product into law firms will be too big of a challenge for a short term goal. Law firms are too risk averse to use new software. Therefore, we have chosen to focus on targeting law students across Canada. 

\section*{The Competition}

Despite considerable searching, we have found no comparable service to what we aim to offer.

The following are the large legal SaaS providers that are at least in the same market space as us, and could be potential buyers of our software.

\begin{quote}
\textbf{QuickLaw} -- QuickLaw is a LexisNexis subsidiary.\footnote{LexisNexis has the world's largest electronic database for legal and public-records related information for legal, risk management, corporate, government, law enforcement, accounting, and academic markets. It has 15,000 employees and operates in over 100 countries, taking in revenue over USD 2.5b in 2011.} It is a database of Canadian cases and legislation. Law schools and law firms hold licences that cost \$10,000 per year. Students at law schools get unlimited access. During their time at law school, students become dependent on it. Law firms, on the other hand, must charge clients \$20 to \$200 per search. So when the addicted students come work at law firms, they end up charging their clients a lot due to inefficient searching habits. It is an effective business model.
\end{quote}

\begin{quote}
\textbf{Westlaw Canada} -- Westlaw Canada is subsidiary of Carswell.\footnote{Carswell is an information services provider for legal, tax \& accounting, and human resource professionals. It is a sub-subsidiary of the Canadian multination giant Thompson Reuter's, which has over USD 13b in yearly revenue.} It is for all intents and purposes the same as QuickLaw, but has better functionality and dependency for legislation. Westlaw creates visual aids for judicial consideration of past cases.
\end{quote}

\begin{quote}
\textbf{CanLII} (Canada Legal Information Institute) -- CanLII provides a database of legislation and cases for free. There is no subscription. CanLII has attempted to provide citations of cases and legislation like we will do, but they are not correct according to the McGill Guide. CanLII is not highly dependable so cannot oust QuickLaw and Westlaw from the market space, despite being free to use.
\end{quote}

\begin{quote}
\textbf{Clio} -- Clio is a Calgary start-up providing document management, time-keeping, and billing services. There are several services like this used by businesses, but Clio is aimed at law firms. 
\end{quote}

\begin{quote}
\textbf{My Legal Briefcase} -- My Legal Briefcase is an embodiment of access to justice by letting non-lawyers do law. It provides standard forms and general document management for beginners. 
\end{quote}

None of these outfits are comparable to our service, though we expect that at least some of them will be used in conjunction with our service. We are building in functionality with CanLII. 

Despite the lack of competition, we hope to continue to add a competitive edge to any services that may be discovered or appear in the future. Our competitive difference will be our design. We are working hard to create an intuitive and aesthetic application. In contrast, Westlaw and QuickLaw, have archaic and ugly user interfaces. Clio and My Legal Briefcase are fairly appealing.


\section*{The Technology}

We plan on delivering our product as SaaS. The product is conveniently split into two parts: the automatic McGill Guide citation formatter, and the interactive precedent tree engine. As our market approach plan is also split into these two parts, I will address them separately.

Before examining each, however, we should explain one key point. Legal research is like a complex of highways and dead-end roads. We want our product to be a car with GPS, whereas everyone else stuck with bicycles. To build the car, we need both an engineer and a cartographer -- i.e. a programmer and a lawyer. Either alone would fail, the lawyer for lack of programming skills, and the programmer for lack of contextual understanding. We make a good pair because David is the lawyer and Stephen is the programmer. Both of us have considerable experience with artistic design.



\subsection*{McGill Guide Citation}

Our initial focus will be on this feature. The McGill guide costs \$50 at a legal book store. Although some aspects of coding it are tricky, there are no processes in it that can only be done as we have done them. Therefore any patent that could be contrived over part of the process could easily be circumvented. 


\subsection*{Precedent Tree Engine}

We have not started coding the precedent tree engine, and, as per below, we likely will not for at least one year.

That being said, we anticipate it will be challenging. But as most software, it will be replicable. \\

The lack of software protection has been a concern to us, but we have a plan to avoid a market takeover from a future competitor. It is connected with our approach to the market, so it will be explained below.

\section*{The Approach to Market}

The following roll-out plan may need more amending than the rest of our ideas in this document.

Please consider the following timeline before continuing.

\begin{table}[htdp]
\begin{center}
\begin{tabular}{ L{3cm} | L{3cm} | L{3cm} }
\textbf{Goal} & \textbf{Date}& \textbf{Purpose}\\
\hline
\hline
Launch free McGill Guide beta for law students & August 15, 2013 & Free testing and market litmus test\\
\hline
Initiate formal McGill Guide testing & September 9, 2013 (when law school resumes) & Bugs and design feedback \\
\hline
Get 6 customers & November 15, 2013 & See if students use it for their first papers\\
\hline
Revamp McGill Guide suite & June, 2014 & Implement selected suggestions from feedback\\
\hline
Begin coding precedent trees & July, 2014 & Begin phase 2\\
\hline 
Launch free precedent tree editor beta for law students & August 15, 2014 & Free testing and market litmus test\\
\hline
Initiate formal McGill Guide testing & September 8, 2014 (when law school resumes) & Bugs and design feedback \\
\hline
Revamp entire website & June, 2015 & Implement good suggestions from feedback on entire website\\
\hline
Begin paid sessions with students & September 7, 2015 & Begin full operations\\
\end{tabular}
\end{center}
\label{Timeline}
\end{table}%


\break


An overriding theme is to hook law students on the service so that they continue with it when they enter the legal workforce. In order to accomplish this, we want to first focus on law students. This will start this fall with the McGill Guide free beta service.

Initially focusing on law students enables us three great advantages: finding bugs, providing design feedback, and testing the market. We want to collect feedback for the 2013-2014 school year.

The following will most likely be rearranged depending on the success of the 2013-2014 year.
In summer 2014, we want to implement the necessary changes to the McGill Guide and create the precedent tree beta service. We will launch the precedent tree beta service in September 2014 and run it throughout the 2014-2015 school year. During this period, the McGill Guide system will continue to be in beta because it is too interconnected with the precedent tree service to test separately.

In the summer of 2015, we will revamp the entire website and furnish it for full-time use. At this juncture, it may be necessary to take on more team-members.

We will then pitch to litigation boutique firms (i.e. firms that focus on legal research), which, at the time, will likely be a mere introduction. They will not yet feel comfortable using a program that has not been paid for or adopted by the market. 

Starting September 2015, we aim to have the full website up for paid service. It is then that we anticipate the longer term business strategy of hooking law students and infiltrating firms will come to fruition.

\section*{Making Money}

Once the service is out of beta, we will sell it through subscription available monthly or yearly. 

Students will get a ``discount'' price, which is really an individual price. A month will cost \$5.95, and a year will cost \$44.95. The yearly cost is the same as about eight months, which is the time students are in school. 

There are 7860 law students in Canada per year. The following table projects revenues according to that number.

\begin{table}[htdp]
\begin{center}
\caption{Revenue from Students}
\begin{tabular}{ R{3cm} |  C{3cm} | L{3cm} }
\textbf{Market uptake} & \textbf{Months Subscribed} & \textbf{Forecast} \\
\hline
\hline
1\% & 1 & \$467.67  \\
3\% & 1 & \$1,403.01  \\
5\% & 1 &  \$2,338.35  \\
10\% & 1 &  \$4,676.70  \\ 
20\% & 1 & \$9,353.40  \\
50\% & 1 &  \$23,383.50  \\ 
100\% & 1 &  \$46,767.00 \\
1\% & 3 & \$1,403.01  \\
3\% & 3 & \$4,209.03  \\
5\% & 3 &  \$7,015.05  \\
10\% & 3 &  \$14,030.10  \\ 
20\% & 3 & \$28,060.20  \\
50\% & 3 &  \$70,150.50  \\ 
100\% & 3 &  \$140,301.00  \\
1\% & 5 & \$2,338.35  \\
3\% & 5 &  \$7,015.05  \\
5\% & 5 &  \$11,691.75  \\ 
10\% & 5 &  \$23,383.50 \\ 
20\% & 5 &  \$46,767.00 \\
50\% & 5 &  \$116,917.50 \\
100\% & 5 &  \$233,835.00 \\
1\% & 12 (1 year) &   \$42,396.84  \\
3\% & 12 (1 year) & \$127,190.52    \\
5\% & 12 (1 year) &  \$211,984.20  \\
10\% & 12 (1 year) &  \$423,968.40  \\ 
20\% & 12 (1 year) & \$847,936.80  \\
50\% & 12 (1 year) &  \$2,119,842.00  \\ 
100\% & 12 (1 year) &  \$4,239,684.00  \\
\end{tabular}
\end{center}
\end{table}%

\break

Firms will get a bulk year-long subscription price according to the population of the firm. We cannot set a price now, as we will need to examine market uptake and how much the program increases productivity, and how much law firms will be willing to pay. But this market could be extremely lucrative, as per Table 2. 

The firm price will be set strategically under the dollar amount that each lawyer is able to save their clients through more efficient researching. As a very conservative estimate, this will be two hours per month per lawyer at a billing cost of \$900. So a yearly subscription could save clients \$21,600 per lawyer. Clearly this amount is far too high for a law firm to pay. From speaking with the director of knowledge management at Norton Rose Fulbright LLP, we anticipate that law firms will not be willing to pay more than \$500 per lawyer per year. Knowing there are about 25,000 practicing lawyers in Canada, The following table projects revenues.

We also intend to allow firms a discounted trial price for 3 months.

\begin{table}[htdp]
\begin{center}
\caption{Revenue from all Lawyers}
\begin{tabular}{ R{3cm} | C{3cm} | L{3cm} }
\textbf{Market uptake} & Months Subscribed & \textbf{Forecast} \\
\hline
\hline
1\% & 12 (1 year) & \$ 124,987.50  \\
3\% & 12 (1 year) & \$ 374,962.50  \\
5\% & 12 (1 year) &  \$ 624,937.50  \\
10\% & 12 (1 year) &  \$ 1,249,875.00  \\ 
20\% & 12 (1 year) & \$2,499,750.00  \\
50\% & 12 (1 year) &  \$6,249,375.00  \\ 
100\% & 12 (1 year) &  \$12,498,750.00  \\
\end{tabular}
\end{center}
\end{table}%

The key to getting multiple law firms as clients is to land a reputable firm first. Once establishing that level of credibility, other firms will feel more comfortable adopting the software.


\section*{Exit Strategy}

There are two strategies for exit.

One is simply not to exit. If the program is up and running, we foresee being able to just let it run and accumulate revenue with biannual marketing regenerative efforts (in September and January, when the law school semesters commence). 

The second strategy is to sell to one of the major legal research outfits identified above. Westlaw or QuickLaw are the two most likely candidates. We need to remember that if they use it, they will have the resources to implement it on a much larger scale that our website. Because they will get more utility out of it, we would increase the selling price above the calculated net present value at time of sale.

\section*{Execution}

We are about a half-way through completing the technical aspects of our service, not including testing. We do not foresee insurmountable problems.


\section*{The Ask}

Although the technical aspects of our website are manageable, we need help with marketing. As this is our first startup, and we do not have much experience in marketing, we are open to all ideas. We intend to optimally minimize marketing costs.

We think a video on the website would really help lure students to the service, the idea being that once a student sees how much time he or she can save by using the service, they would gladly shed a few dollars to do so. The video would ideally be cleanly animated, highly professional, and instructional. We would ask for help with that as well.










\end{document}
